\section{Conclusion}
The Conclusion should reflect on the success of the project, evaluating the effectiveness of using Unity for generating synthetic data. Summarize the results of image generation, randomization, and labeling, and discuss their implications for future work in computer vision. Reflect on the overall experience with Unity, including the advantages of using its Perception package, the ease of integrating 3D models, and the potential for creating realistic and diverse datasets. Conclude by assessing how well the synthetic data generated met the intended objectives and its usefulness for training machine learning models.

\section{What Has Been Done}
In this section, summarize the key objectives and methods used throughout the project. Highlight the steps taken to generate synthetic images using Unity, including the creation of a virtual environment, integration of 3D models, and the implementation of randomization and labeling processes using the Unity Perception Package. Recap the challenges encountered, such as model quality, randomization, and the generation of diverse and realistic scenes. Discuss how these elements contributed to achieving the goals of the project, including generating a dataset for computer vision model training.


\section{Future Work}
In the Future Work section, outline potential improvements and extensions to this project that could enhance the synthetic image generation process. Suggestions could include:

Improving Model Quality: Using higher-quality 3D models and textures to increase the realism of the generated images, which could better simulate real-world scenarios.
Expanding the Virtual Environment: Introducing more complex environments, such as dynamic water simulations, large-scale maritime scenes, or different weather conditions.
Enhanced Randomization: Exploring additional randomizers or more sophisticated randomization techniques to increase the variety in the generated datasets and improve model generalization.
Data Augmentation: Incorporating techniques such as adversarial training or adding noise to synthetic data to further diversify the dataset and improve the robustness of models trained on it.
Evaluation and Comparison: Implementing a comprehensive evaluation framework to compare the performance of models trained on synthetic data versus real-world data, including tasks like object detection, segmentation, or scene recognition.
By identifying these areas, this section should provide a roadmap for future research that could build on the findings of this thesis.

