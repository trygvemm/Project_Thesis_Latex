
This project has shown promising results, demonstrating Unity's potential as a platform for synthetic data generation. Unity proved to be an effective platform for generating synthetic data, thanks to its Perception package, which streamlined image generation, randomization, and labeling. The resulting dataset comprises diverse and well-labeled images that can serve as a foundation for computer vision model training.\\

\noindent The experience with Unity highlighted its advantages, such as the ease of integrating 3D models and its flexibility in creating realistic and diverse datasets. However, the effectiveness of the dataset in training robust computer vision models cannot be fully assessed until model training is completed.


\section{What Has Been Done}
The main accomplishments of this project include: 
\begin{itemize} 
\item Creating a virtual environment in Unity for simulating a harbor docking scenario. 
\item Integrating 3D models representing common maritime objects such as boats, buoys, and docks. 
\item Generating a synthetic dataset using the Unity Perception package, incorporating features such as randomized object placement, lighting, and weather conditions. 
\item Producing labeled data, such as segmentation masks to help training of computer vision models. 
\item Exploring the use of AI tools to enhance the realism of the dataset. 
\end{itemize}
  
\noindent This establish a solid groundwork for exploring the potential of synthetic data in computer vision applications.

\section{Future Work}
Future work will focus on addressing the research question: Is synthetic data sufficient to train a computer vision model for harbor environments, or is real-world data necessary to achieve acceptable performance?\\

\noindent This will involve training various models with the synthetic dataset, collecting real-world data, and testing the model's ability to generalize. It is also interesting to evaluate the impact of combining synthetic and real-world data versus using synthetic data alone.\\

\noindent This work will continue in the spring as part of my master's thesis and is set to be an exciting and challenging next step.
