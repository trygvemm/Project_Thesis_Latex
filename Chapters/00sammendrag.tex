

Denne oppgaven utforsker syntetiske datasett for datasyn-applikasjoner i maritime miljøer, med fokus på havn for autonome fartøyer. En utfordring i dette feltet er den begrensede tilgjengeligheten av mangfoldige virkelige data for å trene datasynsmodeller. Innsamling av data i maritime miljøer kan være kostbart og tidkrevende, og kan kreve spesialisert utstyr og manuell annotering. syntetiske datasett kan være et alternativ for disse utfordringene. Ved å bruke Unitys Perception-pakke, sammen med egendefinerte randomisatorer, ble et syntetisk datasett på 2350 bilder utviklet. Datasettet inneholder vanlige maritime objekter som båter, kajakker og bøyer under varierende miljøforhold.\\

\noindent Prosjektet demonstrerer flere fordeler med syntetiske data, inkludert reduserte kostnader, økt skalerbarhet og muligheten til å generere store datasett. Det tar for seg utfordringene knyttet til forskjellene mellom syntetiske og virkelige data. Ved å implementere teknikker som AI-drevne forbedringer og randomisering, kan realismen og mangfoldet i datasettet forbedres, noe som kan gjøre det mer egnet for virkelige applikasjoner, som autonome fartøyer i havnemiljøer.