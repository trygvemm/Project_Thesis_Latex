
This thesis explores the creation of synthetic datasets for computer vision applications in maritime environments, with a focus on harbor scenarios for autonomous vessels. A key challenge in this field is the limited availability of diverse and high-quality real-world data for training computer vision models. Collecting data in maritime environments is often expensive and time-consuming, requiring specialized equipment and manual annotation. To address these challenges, synthetic datasets can be an alternative. Using Unity’s Perception package, along with custom randomizers, a synthetic dataset of 2,350 labeled images was created. The dataset includes common maritime objects such as boats, kayaks and buoys, and covers a broad range of varied environmental conditions, including changes in weather, lighting and camera angles.\\

\noindent The project demonstrates several benefits of synthetic data, including reduced costs, increased scalability and the ability to generate large datasets. It also addresses the challenges associated with the domain gap between synthetic and real-world data. By implementing techniques such as AI-driven enhancements and randomization, the realism and diversity of the dataset are improved, making it more suitable for real-world applications, such as autonomous vessels in harbor environments.
